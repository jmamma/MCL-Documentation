\chapter{Key Concepts}

\begin{itemize}
\item \textbf{Page:}
\\
The MCL firmware consists of pages accessible through the MC's GUI. Each page contains distinct functionality and is described in this manual.
\item \textbf{Project:}
\\
A project stored on the Micro SD-Card.
The maximum number of projects is only limited by the SD Card capacity.

\item \textbf{Grid:}
\\
The MCL Firmware uses a Grid/Slot system to store Tracks.\\
Each project contains two grids, A and B. Grid dimensions are 16 Slots x 128 Rows.\\
\\
Grid A is used to store 16 MD tracks.\\Grid B is used to store 6 External MIDI tracks + 4 AUX tracks.

\item \textbf{Row/Pattern:}
\\
A row of the Grid.

\item \textbf{Slot:}
\\
A position in the Grid where a Track can be stored. (Either occupied or unoccupied).

\item \textbf{Track:}
\\
An internal sequencer track that may contain both sound and MIDI sequencer data.

There are 3 types of tracks.
\begin{itemize}

\item \textbf{MachineDrum Track} (Grid A: Slots 0-15):
A sequencer track for the Elektron MD. Each MD Track contains the Machine's Sound Settings and Sequencer Data.

\item \textbf{External MIDI Track} (Grid B: Slots 0-5):
A polyphonic sequencer track used to control a sound module connected via MIDI. Each External MIDI track contains Sequencer Data, and for supported Elektron devices sound data is retained. 

\item \textbf{AUX Track} (Grid B: Slots: 12-15)\\
An auxiliary track. Used to store/recall the Machinedrum's master FX settings, LFO settings, audio Routing and Tempo settings.
\end{itemize}

\end{itemize}

