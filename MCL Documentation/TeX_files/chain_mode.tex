\chapter{Chain Mode:}
Chain mode was inspired by old school music tracker software that generate music by iterating through sequences and sounds stored in a vertical column.
\section{How:}
Each slot in the Grid has the ability to play for N loops and then jump to another slot in a specific row of the same column.  These values are specified by the LOOPS and ROW settings in the SlotMenu and apply to the current slot. Jumping between slots is referred to as a transition. Just before a transition occurs, Machine settings are sent to the MD and the internal sequencer data for that track is loaded. All 20 tracks can be configured to transition at the same time.\\
 \\
Chain mode will not send pattern data to the MD. Therefor you must perform all your sequencing using MCL's internal sequencer. For your convenience it is possible to merge a slot's MD sequence data from within the Save Page.
\section{Modes}
\textit{Chain Mode behaviour can be changed in the SlotMenu or GlobalSettings-->ChainMode menu by setting ChainMode to one of 3 three settings: Automatic, Manual and Random.}

\begin{itemize}
	\item Automatic: If the number of loops is greater than 0, slots will automatically jump to the specified Row after N loops.
	\item Manual: Automatic slot jumping is disabled, but tracks can be chained using the quantization rules in the Chain Page.
	\item Random: Slots will jump after a random number of iterations to a random row position bounded by the min and max settings specified in GlobalSettings-->Chain Mode
\end{itemize}


