\chapter{Modes of Operation}
MCL features two modes of operation: Write Mode and Chain Mode. It is important that you understand how these modes differ, as they affect how MCL loads and transfers sequencer + machine data to your MD.\\
\\
Tracks are either Written, or Chained, and are selected using the Trigger Interface from either the Write or Chain Pages respectively. \\
\\
\textit{Chain mode is enabled through the SlotMenu or GlobalSettings-->Chain menu.}

\section{Write Mode vs Chain Mode}
\textbf{Write mode} is useful for transferring kit or pattern data to and from the MC. \\
\\
Write mode works by automatically receiving current pattern and kit dumps from the MD. The MC then inserts the selected track(s) in to the pattern/kit and transfers them back. Once a transfer is complete, the internal sequencer data for the selected tracks is loaded on the MC and tracks are un-muted as per quantization settings.\\
\\
Sending pattern data to the MD is relatively slow, so you will generally need to use quantization settings to avoid hearing audible artefacts.\\
\\
\textbf{Chain mode}  on the other hand loads sequencer and machine data seamlessly and allows individual tracks to be linked together to form complex musical phrases. \\
\\
Chain mode does not transfer pattern data to the MD. It instead  loads up the internal sequencer data for the selected track(s) on the MC. Machine settings for tracks are transferred using SYSEX and CC messges (not kit dumps) and are interleaved between 16th notes of the step sequencer. This allows for seamless loading of tracks and their corresponding sequences. Loaded tracks can be made to automatically transition to a slot on another row without any audible delay. \\
\\
Because chain mode does not send the MD sequencer data to the MD, you must rely on the MC's internal sequencing capabilities. You can choose to merge a slot's MD sequencer data in to the internal sequencer data. When this is done and a track is writtern, the MC will sequence the original MD pattern. The resulting sequence will be indistinguishable from the original.

\subsection{Quantization differences:}
Quantization settings work differently in each mode.\\
\\
 In Write mode tracks are un-muted at the step next corresponding to the quantization setting. \\
\\
Chain Mode will schedule a track transition at the step next corresponding to the quantization setting, and the track's sequence will begin from step 0 at the transition step.
 
