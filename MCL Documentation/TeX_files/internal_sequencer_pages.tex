\chapter{Internal Sequencer Pages:}

\textit{The primary Sequencer Pages are accessed using the \textbf{PageSelect} page. These include:
\begin{itemize}
    \item (MD) Step Edit Page
    \item (MD) Parameter Edit Pages
    \item (Ext MIDI) PianoRoll Page
    \item (MD/Ext MIDI) Chromatic Page
\end{itemize}}
\section{Track selection}
For the Step Edit and Parameter Edit pages, the current track selection is synced to the Machinedrum. When you change track on the MD the track will change on the MegaCommand.\\\\
For the PianoRoll page, the current track can be selected from the PianoRoll menu's TRACK SELECT option. Alternatively, if an external MIDI device is connected to port2, the track will automatically change to one that is set to the same MIDI channel as incoming note data.
\\\\
\textit{Automatic track select can be disabled from the MCL system menu option Global -> Machinedrum -> TRACK SELECT}
\section{Live Record:}
Live Record mode can be activated from any Sequencer page by pressing \textbf{[ Save ]}.\\Depending upon the page type, live Record can be used to record:
\begin{itemize}
    \item Trig presses
    \item Parameter changes, CC Automation
    \item Notes played in Chromatic Mode
    \item Notes played on the PianoRoll page.
\end{itemize}
Pressing \textbf{[ Load ]} during a recording will clear the sequence for the current track.

\newpage
\section{Track menu}
\screenshot{track_menu.png}

The track menu will be opened when holding \textbf{[ Shift 2 ]}, and the entry activated on release, similar to the slot menu.
The track menu consists of the following entries that are common to all Sequencer Pages:

\begin{figure}[hb]
    \begin{tabular}{|l|l|}
    \hline
    \rowcolor[HTML]{C0C0C0} 
    Entry            & Function                                                        \\ \hline
    Edit     & change editor mode   \\ \hline
    Track Select     & select active track                  *only visible when TRACK SELECT = MAN                                              \\ \hline
    Copy             & \begin{tabular}[c]{@{}l@{}}TRK: copy track, ALL: copy pattern\end{tabular}                                             \\ \hline
    Clear            & \begin{tabular}[c]{@{}l@{}}TRK: clear track, ALL: clear pattern\end{tabular}                                           \\ \hline
    Paste            & \begin{tabular}[c]{@{}l@{}}TRK: paste track, ALL: paste pattern\end{tabular}                                           \\ \hline
    Speed & \begin{tabular}[c]{@{}l@{}}The current track's playback speed\\ 1x, 2x, 3/2x, 3/4x, 1/2x, 1/4x, 1/8x.\\ Hold [Write] and release [Shift 2] to change speed of all tracks. \end{tabular}                          \\ \hline
    Shift            & \begin{tabular}[c]{@{}l@{}}L/R: shifts the track left/right.\\ L ALL/R ALL: shifts the pattern left/right.\end{tabular} \\ \hline
    Reverse          & TRK: reverse the track, ALL: reverse the pattern                                                                          \\ \hline
    \end{tabular}
\end{figure}